\documentclass[jcp,aip,amsmath]{revtex4-1}
\usepackage{xcolor,array}
\usepackage{color}
\usepackage{txfonts}

\newcommand{\half}{\textstyle \frac{1}{2}}
\newcommand{\quarter}{\textstyle \frac{1}{4}}


% note, to complile, do something like:
%#!/bin/bash
%file=$1
%rm ``$file''.aux
%latex ``$file''.tex
%bibtex ''$file"
%latex ``$file".tex
%latex ''$file".tex
%dvips -Ppdf -o ``$file".ps "$file".dvi
%ps2pdf ``$file".ps
%echo  `` done compiling $file''
%mupdf "$file.pdf" &



\begin{document}



\title{Evolve Equations}
\author{Matthew K. MacLeod}
\maketitle
\today

\section{Maths.ex}
The code for following equations is located in lib/evolve/maths.ex

\subsection{Linear Algebra}
dot product for two vectors,
\begin{align}
\mathbf{A} \cdot \mathbf{B} = A^\dag B = \sum_i^n A_i * B_i
\end{align}

\subsection{Statistics}

standard deviation, $\sigma$, is defined as, the square root of the variance
\begin{align}
\sigma = \sqrt{v} =  \sqrt{\frac{1}{n}\sum_i^n (x_i - \mu)^2}
\end{align}
where the mean, $\mu$, is defined as
\begin{align}
\mu = \frac{1}{n} \sum_i^n x_i
\end{align}

\subsection{Probability}
 The normal probability distribution (normal\_pdf in code), the famous bell-shaped curve is
defined as
\begin{align}
f(x|\mu,\sigma) = \frac{1}{\sqrt{2\pi\sigma}}\mathrm{exp}\left(-\frac{(x-\mu)^2}{2\sigma^2}\right)
\end{align}
and $\mu$ is again the mean.

The beta distribution can be used for Bayesian inference (conditional probabilities) and is defined
\begin{align}
\mathrm{beta\_pdf} = \frac{x^{\alpha-1}(1-x)^{\beta-1}}{B(\alpha,\beta}
\end{align}

where $B(\alpha,\beta)$ is a normalization constant to ensure that the total probability integrates to 1.
and is
\begin{align}
B(\alpha,\beta) = \frac{\Gamma(\alpha) \Gamma(\beta) }{\Gamma(\alpha + \beta)}
\end{align}
This function uses the gamma function, which is the generalization of a factorial, and which is approximated in the code using Windschitl reworking of Stirlings as follows
\begin{align}
\Gamma(t) = \int_0^{\infty} x^{t-1} e^{-x} dx \approx  \sqrt{\frac{2\pi}{t}}\left(\frac{1}{e}\left(t + \frac{1}{12t -\frac{1}{10t}}\right)\right)^t
\end{align}

Note, especially for easy testing, positive integers should follow
\begin{align}
\Gamma(n)=(n-1)!
\end{align}





\subsection{Machine Learning}


\end{document}
